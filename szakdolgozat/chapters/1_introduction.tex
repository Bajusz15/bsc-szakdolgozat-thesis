\Chapter{Bevezetés}

% A fejezet célja, hogy a feladatkiírásnál kicsit részletesebben bemutassa, hogy miről fog szólni a dolgozat.
% Érdemes azt részletezni benne, hogy milyen aktuális, érdekes és nehéz probléma megoldására vállalkozik a dolgozat.
\section{Probléma megfogalmazása}
Manapság a vásárlások nagyrésze online történik, a termékeket pedig házhoz szállítják futárral.
A robotika fejlődése miatt, mint ahogy sok más egyszerű munka, a csomag szállítás is automatizálódni fog, drónok által.
Egy ilyen rendszer nagyon sok adatot generál, amit fel kell dolgozni valós időben és eltárolni az adatokat.
Az elmúlt évtizedben olyan ekommersz óriások mint az Amazon már próbálkoztak több megoldással drón szállítás terén, így a közeljövőben elképzelhető, hogy egy nagyon valós probléma lesz a drónokat irányító, figyelő rendszer működtetése.

\Section{A szakdolgozat célja}
Egy olyan rendszert szeretnék bemutatni, ami gyors, hatékony, képes konkunkurrens működésre elosztott rendszerként a felhőben és valós időben tudja szimulálni a drónok működését, telemetria adatainak feldolgozását.
A rendszer több alkalmazásból épül fel, az alkalmazások Go (Golang) nyelven készültek és Docker konténerekben futnak.
A rendszer modulárisan épül fel, így különböző komponenseket kicserélhetjük, hogy összehasonlíthassuk a rendszer hatékonyságát a különböző konfigurációkkal.
Összehasonlításra kerül relációs és dokumentum alapú adatbázisok teljesítménye, valamint olyan internetes kommunikációra használt protokollok és formátumok hatékonysága mint a HTTP/2 gRPC, HTTP/1.1 JSON.
De olyan érdekességekről is lesz szó, mint hogy a Földön 2 pont között nem az egyenes a legrövidebb út.



% Ez egy egy-két oldalas leírás.
% Nem kellenek bele külön szakaszok (section-ök).
% Az irodalmi háttérbe, a probléma részleteibe csak a következő fejezetben kell belemenni.
% Itt az olvasó kedvét kell meghozni a dolgozat többi részéhez.
