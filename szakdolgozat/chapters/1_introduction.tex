\Chapter{Bevezetés}

% A fejezet célja, hogy a feladatkiírásnál kicsit részletesebben bemutassa, hogy miről fog szólni a dolgozat.
% Érdemes azt részletezni benne, hogy milyen aktuális, érdekes és nehéz probléma megoldására vállalkozik a dolgozat.
Manapság a vásárlások egyre nagyobb része történik online, a termékeket pedig házhoz szállítják futárral.
A robotika fejlődése miatt, mint ahogy sok más egyszerű munka, a csomagszállítás is automatizálódni fog, melyre drónok is bevethetők.
Egy ilyen rendszer nagyon sok adatot generál, amit fel kell dolgozni valós időben és eltárolni.
Az elmúlt évtizedben olyan ekommersz óriások, mint az Amazon már próbálkoztak több megoldással drón szállítás terén, így a közeljövőben elképzelhető, hogy egy nagyon valós probléma lesz a drónokat irányító, ellenőrző rendszer működtetése.
Egy olyan rendszert szeretnék bemutatni, ami gyors, hatékony, képes konkurrens működésre elosztott rendszerként a felhőben, valós időben tudja szimulálni a drónok működését, telemetria adatainak feldolgozását.
A rendszer több alkalmazásból épül fel. Az alkalmazások Go (Golang) nyelven készültek és Docker konténerekben futnak.
A rendszer modulárisan épül fel, így különböző komponenseket kicserélhetjük, hogy összehasonlíthassuk a rendszer hatékonyságát a különböző konfigurációkkal.
Összehasonlításra kerül relációs és dokumentum alapú adatbázisok teljesítménye, valamint olyan internetes kommunikációra használt protokollok és formátumok hatékonysága, mint a HTTP/2 gRPC, HTTP/1.1 JSON.
Olyan érdekességekről is lesz szó, mint hogy a Föld felszínén 2 pont között nem az egyenes a legrövidebb út.



% Ez egy egy-két oldalas leírás.
% Nem kellenek bele külön szakaszok (section-ök).
% Az irodalmi háttérbe, a probléma részleteibe csak a következő fejezetben kell belemenni.
% Itt az olvasó kedvét kell meghozni a dolgozat többi részéhez.
