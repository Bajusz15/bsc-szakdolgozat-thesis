\Chapter{Összefoglalás}

Összességében meg vagyok elégedve az eredményekkel.
A rendszer moduláris felépítése jól sikerült, az összehasonlítások tökéletesen működtek és érdekes eredményeket produkáltak, de persze semmi meglepő nincs az eredményekben.
\par Számomra nagyon érdekesek a különböző architektúrák, új kommmunikációs protokollok és adatbázisok által nyújtott előnyök.
Amit hiányolok a programból, de sajnos nem maradt idő az implementálásra az egy valós idejű Apache Kafka adatcsatorna.
Az Apache Kafka egyfajta üzenet bróker, amely képes nagy forgalmú hiba toleráns működésre.
Mivel a telemetria adatokra tekinthetünk úgy, mint kritikus adatok, egy ilyen modullal ki lehetne egészíteni a programot.
A drónok szimulációja lehetett volna érdekesebb, ha tudunk valós térképek alapján útvonalakat optimalizálni.
A rendszer jövőjét mindenképpen abban látom, hogy több protokollt és adatbázist lehet összehasonlítani.
Érdekes eredmények születhetnének MQTT protokollal, és más dokumentum alapú vagy nagyon sok írásra tervezett adatbázisokkal.

