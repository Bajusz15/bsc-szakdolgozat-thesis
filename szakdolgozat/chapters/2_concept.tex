\Chapter{Koncepció}

\Section{Hasonló rendszerek, megoldások áttekintése}

Az Amazonnak már működő szállítási rendszere van, mely drónnal 30 percen belül eljuttatja a 2,3 kg-nál kisebb tömegű csomagokat az ügyfelekhez.
Ezek a drónok csak 24 km-es hatótávval rendelkeznek, így csak akkor biztonságos egy kézbesítés ha a célállomás maximum 12 km-re van.
A csomag maximum térfogatáról nincsenek pontos adatok.
Ezt a rendszert \textit{Prime Air}-nek hívják \cite{prime-air}. Az Amazon Prime Air egy drónja látható \aref{fig:prime}. ábrán.

\begin{figure}[h]
    \centering
    \includegraphics[scale=0.4]{images/prime.jpg}
    \caption{Amazon Prime Air drón}
    \label{fig:prime}
\end{figure}

Először 2016-ban próbálták ki, azóta még nem sikerült bevezetni különböző jogi korlátozások miatt. Olyan jogi elvárásoknak kell megfelelni, mint például az alábbiak.
\begin{itemize}
    \item Muszáj egy drón pilótának ellenőrizni a repülést és szükség esetén kötelező közbeavatkoznia.
    \item Egy ilyen pilóta csak 1 drónt felügyelhet egyszerre, kivéve ha rendelkezik több felügyeletéhez megfelelő engedéllyel.
    \item A drónokat nem lehet közvetlen egy személy vagy gépjármű fellett működtetni.
    \item Problémát jelent továbba, hogy nem minden légtér megfelelő a drónokkal való szállításra. A reptér mellett fekvő nagyvárosokban például biztos hogy nem lehet drónokkal szállítani.
    \item A csomagot szállító drónnak is kell rendelkeznie engedéllyel, hogy alkalmas kereskedelmi szállításra.
\end{itemize}

Az FAA 2020-ban engedélyezte az Amazonnak a drónokkal való csomagszállítást, de egyelőre csak tesztelik a technológiát pár városban.A jövőben teljesen elképzelhető, hogy a csomagok felét drónok szállítják ki. Önmagában az Amazon naponta átlagosan 7 millió csomagot szállít ki, és bevallásuk szerint a csomagok 86\%-a drónnal szállítható lenne.

Az Amazon felhőszolgáltásai között is található olyan szolgáltatás, amellyel könnyedén kezelhetjük az IoT eszközöket, jelen esetben a drónokat.
Ilyen az \textit{IoT Core} vagy az \textit{IoT Things Graph} amin grafikus felületen összeköthetjük és beállíthatjuk az eszközeink közötti kommunikációt.
A DHL is próbálkozott hasonló csomagkézbesítő megoldásokkal. Ők egy úgynevezett  \textit{Parcelcopter}-el szállították a csomagokat (\ref{fig:parcelcopter}. ábra).

\begin{figure}[h]
    \centering
    \includegraphics[scale=1.0]{images/parcelcopter.jpeg}
    \caption{DHL Parcelcopter V4.0}
    \label{fig:parcelcopter}
\end{figure}

A DHL Percelcopter alapvetően más problémát akar megoldani, a nagyon nehezen elérhető helyekre probálnak csomagokat szállítani.

A \textit{FlytBase} \cite{flyt} vállalat már kínál szoftveres megoldásokat drónok automatizációjára és valós idejű figyelésére és irányítására, útvonal tervezésre valamint flotta menedzselésre.
Már van kész megoldásuk a következőkre:
\begin{itemize}
    \item Biztonsági megfigyelések
    \item Automatikus dokkolás állomásokhoz
    \item Veszélyhelyzetre reagálás, ez lehet vér, szerv vagy gyógyszer szállítás.
    \item Csomag szállítás
\end{itemize}

Elterjedt termékük a FlytNow, ami úgy működik, hogy a drónunkat összekötjük egy appon keresztül a felhőben lévő FlytNow rendszerrel,
majd interneteN keresztül elérünk egy verzérélőpultot \ref{fig:flytdash} amin irányíthatjuk a drónjainkat és láthatjhuk az adatokat.
Ez a rendszer arra összpontosít hogy gyorsan, bonyolult telepítés és beállítás nélkül tudjunk drónokat irányítani.
Ezt a rendszer magánembereknek és cégeknek egyaránt elérhető, több csomagban.
Van egy másik FlytWare nevű termékük raktár menedzselésre és logisztikai feladatokra. Olyan feladatokra specializálódnak a drónok mint a bárkód szkennelés,
zárt környezetben repülés. Természetesen a raktár adatait elérjük a rendszer vezérlőpultján. Így például egy leltár sokkal egyszerűbben megoldható.


\begin{figure}[h]
    \centering
    \includegraphics[scale=0.3]{images/flyt-dashboard.png}
    \caption{Flytnow livestream dashboard}
    \label{fig:flytdash}
\end{figure}

Az \textit{ANRA Techonlogies}\cite{anra} cég is foglalkozik mindenféle drón irányítás, megfigyeléssel. A SmartSkies néven futó rendszere például drónokat használ mindenféle szenzorral,
ahhoz hogy a pontosabb adatokat kapjunk megfigyeléses mérésekhez. Például egy olyan légtérben ahol rengeteg drón repül előfordulhatnak hibák, egy drón rossz adatokat küld helyzetéről vagy
teljesen megszűnik a kommunikáció, esetleg idegen tárgyak akadályozzák a légtérben  való forgalmat, stb.
Ebben a rendszerben minden drón el van látva radarral, kamerával és más szenzorokkal. Így az összegyűjtött adatokat kombinálva
egy sokkal pontosabb képet kapunk ami hasznos a navigációhoz, és akár fél percre előre meg tudjuk mondani mi lesz az adatok alapján, előre jelezve az esetleges hibákat.


A \textit{BeeCluster}\cite{beecluster} egy nyílt forráskódú, Golang-ban írt drón orchesztrációs rendszer. Előnye, hogy prediktív optimizációt használ.
Tehát a feladatok előtt egy eszimulációt futtat, ami alapján optimalizálja, azaz hatékonyabbá és gyorsabbá teszi a valós folyamatot.
Olyan feladaotkra használják mint a földrajzi térképezés, wifi lefedettség térkép, új utak feltérképezése.


Jelenleg nincs a piacon olyan, \textbf{bárki számára elérhető} szállításra gyártott drón ami hatékonyan szállít csomagot. Egyelőre csak a szállításban
érdekelt cégek használnak ilyen drónokat, de sajnos nem osztják meg a pontos paramétereket. Annyi használható
információ érhető el, hogy az Amazon teszt drónjai átlag 70 km/h sebességgel 12 kilométeres körzetben tudnak szállítani
2,3 kg-tól könnyebb csomagokat.
De nem tudjuk a drón súlyát és azt se, hogy mennyi üzemanyagot fogyaszt.

A Missouri Egyetemen végeztek egy \textit{a drónok fogyasztására irányuló kutatást}\cite{dron-szallitas-kutatas},
amiből több érdekes dolog is kiderül. Mind a kis (1-3 kg) és nagy (5-15kg) drónok hatékony sebessége  12m/s körül van.

\begin{figure}[h]
    \centering
    \includegraphics[scale=0.4]{images/fogyasztas-kis-dronok.png}
    \caption{Kis drónok fogyasztása}
    \label{fig:kis}
\end{figure}


\begin{figure}[h]
    \centering
    \includegraphics[scale=0.4]{images/fogyasztas-nagy-dronok.png}
    \caption{Nagy drónok fogyasztása}
    \label{fig:nagy}
\end{figure}
%TODO: Ide még irni, hogy mi alapjan lehetne optimalizalni, mennyisegeket hogy mennyi adat keletkezik.

\Chapter{Szállítási probléma}
\Section{A szállítási probléma absztrakt modellje}

A szállítási modellünk alapvetően úgy nézne ki, hogy a raktár és a kiszállítási hely közti távolság függvényében a csomagokat a raktárból drónok, vagy drónokkal felszerelt teherautók viszik ki.
A teherautók a kiszállítási hely közelében használnák a drónokat hogy kézbesítsék a csomagokat.

Feltételezzük, hogy a drón és a csomag ugyanazon a raktárba vagy tehergépkocsiba található, de a szimuláció szemszögéből nem fontos hogy raktár vagy tehergépkocsi, az a lényeges, hogy egy helyen található, és a drón automatikusan fel tudja venni az adott csomagot
valamilyen rendszeren keresztül.

A drónok 3G, 4G vagy 5G hálózaton kommunikálnának az adatközpontokkal.
Az adott hálózaton elérhető sávszélesség függvényében a drónok változtathatják az elküldött adatok mennyiségét és a küldés ütemezését, gyakoriságát.

A drónok repülése előre megtervezett. A szállítási feladat egy speciális esetét feltételezzük, a hozzárendelési feladatot.
Tehát hogy egy drónhoz 1 csomagot rendelünk. A kis hatótáv és erőforrások hiánya miatt a drónok egyébként sem tudnának egynél több csomagot szállítani csak nagyon ritka esetben.
Azt, hogy melyik drónhoz melyik csomag tartozik a hozzárendelési feladat optimalizálásával kapjuk meg.
A drónokból a szállítás során keletkező nagymennyiségű adatot az adatközpontoknak valós időben, hibamentesen
és hatékonyan kell tudnia kezelni anomáliák nélkül.
Ehhez nagyon fontos a terhelés megfelelő elosztása.

A drónok csomag szállítási folyamatáról a következőket feltételezzük:
\begin{itemize}
    \item Egy drón  véges akkumulátoridővel vagy üzemanyaggal rendelkezik és mindig vissza kell tudnia térnie egy töltőállomásra, amely lehet egy drónszállító teherautó vagy raktár.
    \item A csomag kézbesítésnek a csomag átvételére vonatkozó ideje elhanyagolható.
    \item Minden drón fel van szerelve megfelelő kommunikácós és helymeghatározó eszközökkel, valamint kommunikál az adatközpontokkal.
    \item Feltételezzük hogy a drónok el vannak látva szenzorokkal hogy felismerjék az akadályokat és erről értesítsék az adatközpontokat.
    \item Egy drón csak 1 csomagot képes szállítani.
\end{itemize}

A mi esetünkben csak akkumlátorral üzemelő drónok lesznek, és a drónok fogyasztása előre ismert.
Minden drón küld telemetria adatokat.
A drónok a telemetria adatok részeként küldenek helymeghatérozásra alkalmas földrajzi szélesség, hosszúság koordinátákat és magasságot.
Továbbá, kapunk még információkat a drón sebességéről,gyorsulásáról, tájoló iránytűjének állásáról, motorjainak hőmérsékletéről,
akkumulátor(ok) töltöttségről és hőmérsékléről, mindezt egy időpecséttel megbélyegezve hogy az üzenetek elküldésének idejét is tudjuk.
Hogy el tudjuk dönteni melyik drón hova tud a leghatékonyabban szállítani, a következő mennyiségeket jelölni kell. Hány km-re van a felszállási ponttól a csomag leadási helye (távolság km-ben), mekkora a drón fogyasztása egy adott csomag súlyával együtt.
Mivel minden drónhoz csak 1 csomagot rendelünk, ez a szállítási feladat egy speciális fajtája, úgynevezett hozzárendelési feladat amit a \textit{magyar módszer algoritmussal}\cite{magyar-modszer} meg lehet oldani.
Tehát ha ismerjük a drónokat és a csomagokat, minden drónhoz rendelhetünk egy képletet ami megmutatja a fogyasztási költséget az adott csomaggal. Mivel ismerjük a drónok felszállási helyét és a csomagok címzett helyét is, jelölni tudjuk a távolságot is.
A távolság és fogyasztásból kijön egy adott szállítás költsége. Értelemszerűen az adott szállításhoz azt a drónt keressük amelyik minimális költséggel szállítja el a csomagot. Tehát a fogyasztást szeretnénk minimalizálni.
Persze ezeket lehetne még bonyolítani, tiltási tényezőket használni, például hogy egy drón maximum milyen súlyú csomagot vihet.

\paragraph{Konkrét példa}
Egy optimalizálási feladat a következőképp nézhet ki. Az egyszerűség kedvéért tegyük fel, hogy 4 drón és 4 csomag kapcsolatát vizsgáljuk. Jelölje $d_{i}$ a drónokat és $c_{j}$ a csomagokat, $i, j = 1, 2, 3.$
A feladat modellje a következő:

A drónok fogyasztása távolság egységenként rendre:
$d_{1}$ fogyasztása = 800
$d_{2}$ fogyasztása = 700
$d_{3}$ fogyasztása = 650
$d_{4}$ fogyasztása = 390

A csomagok súlyával most az egyszerűség kedvéért nem számolunk, csak a távolságot vesszük figyelembe.
$c_{1}$ távolsága = 0,749
$c_{2}$ távolsága = 1,255
$c_{3}$ távolsága = 2,269
$c_{4}$ távolsága = 1,844 \\

A költségeket a következő képlettel számolhatjuk ki.
$K\ddot olts\acute eg_{ij} = d_{i} * c_{j}$.
A kapott költségek a következő táblázatban \ref{tab:optim} figyelhetőek meg.

\begin{table}[h]
    \centering
    \caption{Optimalizálási feladat táblázat. Az adott csomag szállításának költsége az adott drónnal.}
    \label{tab:optim}
    \begin{tabular}{c|c|c|c|c|}
        \diagbox[width=10em]{Csomag}{Drón}& d1 & d2 & d3 & d4\\
        \hline
        c1 & 599.92 & 524.93 & 487.43 & 292.46 \\
        c2 & 1004.19 & 878.67 & 815.9 & 489.54 \\
        c3 & 1815.68 & 1588.72 & 1475.24 & 885.14 \\
        c4 & 1475.76 & 1291.29 & 1199.1 & 719.43 \\
        \hline
    \end{tabular}
\end{table}


\begin{gather*}
    599.92x_{11} +524.93x_{12} +487.43x_{13} +292.46x_{14} +1004.19x_{21} +878.67x_{22} +{}\\
    815.9x_{23} +489.54x_{24} +1815.68x_{31} +1588.72x_{32} +1475.24x_{33} +885.14x_{34} +{}\\
    1475.76x_{41} +1291.29x_{42} +1199.1x_{43} +719.43x_{44} \rightarrow min \\
\end{gather*}

A feladat megoldását kezdjük a költségmátrix felírásával, ami lényegében megegyezik az előző táblázattal.

\[
    \begin{matrix}
        & D1 & D2 & D3 & D4 \\
        C1 & 599.92 & 524.93 & 487.43 & 292.46  \\
        C2 & 1004.19 & 878.67 & 815.9 & 489.54\\
        C3 & 1815.68 & 1588.72 & 1475.24 & 885.14 \\
        C4 & 1475.76 & 1291.29 & 1199.1 & 719.43 \\
    \end{matrix}
\]

Mivel a feladatban a kereslet és kínálat egyensúlyban van, alkalmazhatjuk a magyar módszer algoritmus következő lépéseit. A kapott végeredmény az, hogy a minimális költség 3562.83.
A következő mátrix már csak a megoldást tartalmazza az eredeti költségmátrixra vetítve:

\[
    \begin{matrix}
        & D1 & D2 & D3 & D4 \\
        C1 & 599.92 &  &  &  \\
        C2 &  & 878.67 & & \\
        C3 &  &  &  & 885.14 \\
        C4 &  &  & 1199.1 &  \\
    \end{matrix}
\]


\Section{A szállítási probléma egy konkrét példája}

A szállítás úgy kezdődik, hogy az adatközpont lekérdezi a szállításra váró csomagokat. Ezután lekérdezi az összes szabad drónt.
Elemzi a drónok és a csomagok paraméterei alapján, hogy hogyan lenne a legoptimálisabb a szállítás. Ez valamilyen eredményt generál, ami
alapján az adatközpont a csomagokhoz drónokat rendel a megfelelő útvonallal. A drónok elhagyják a raktárat a csomaggal, és folyamatosan küldik az adatokat állapotukról,
az adatközpont pedig elemzi és feldolgozza ezeket.

Pédaként egy drón felszáll egy csomaggal, a csomagban lévő tárgy kevesebb mint 1kg. A drón a kijelölt útvonalon halad, nem adódik semmilyen probléma a cél
elérése közben. A drón leadja a csomagot a kijelölt helyen, és jelez az adatközpontnak hogy sikeresen kézbesítette a csomagot.
Ekkor az adatközpont jelez a megrendelőnek hogy a csomagját átveheti. A drón felemelkedik és visszatér a raktárba, a kijelölt útvonalon. Itt az akkumulátorját szükség esetén
feltöltik, majd a drón felveheti a következő csomagot.

Egy másik példaként egy drón felszáll egy csomaggal, a csomagban lévő tárgy kevesebb mint 1kg. A drón a szállítás közben különösen magas hálózat interferenciát érzékel,
és hibajelet küld az adatközpontnak. Az adatközpont az elemzés szerint arra jut hogy az interferencia nem természetes,
túl nagy az esélye hogy a drónt szándékosan zavarják. Az adatközpont jelez a drónnak, hogy a szállítást szakítsa meg, térjen vissza a raktárba.
A drónnak a hiba miatt a legközelebbi repülés előtt több diagnosztikai teszten át kell mennie, hogy megbizonyosodjunk az drón üzemképes működéséről.
%TODO
% TODO: Készíteni kellene egy sematikus ábrát, amin egy szállítási állapot, gráfos formában megadott probléma látható.


\Section{A szállításhoz kapcsolódó számítási problémák}
A szállítás pontos szimulációjához a Föld alakját is figyelembe kell venni a repülésnél.
Ugyanis itt a Cartesian féle geometria nem jól működik, mivel a Földnek gömb alakja van.
Emiatt teljesen máshogy kell számolni távolságot, mint a Cartesian féle geometriában.
\subsection{Cartesian szerinti}
\begin{gather}
    P_1(5,6,3) \\
    P_2(7,4,9) \\
    V = 10 \frac{m}{s}
\end{gather}


Kiszamitas:

\begin{gather}
    x = r * cos(\phi) * sin(\theta) \\
    y = r * sin(\phi) * sin(\theta) \\
    z = r * cos(\theta)
\end{gather}


X, Y, Z, r

\begin{gather}
    x = X_2 - X_1 = 7 - 5 = 2 \\
    y = Y_2 - Y_1 = 4 - 6 = -2 \\
    z = Z_2 - 7_1 = 9 - 3 = 6 \\
    r = \sqrt{2^2 + (-2)^2 + 6^2} = \sqrt{44}
\end{gather}

$
cos(\phi) sin(\phi) cos(\theta) sin(\theta)
$

\begin{gather}
    cos(\phi) = \frac{x}{\sqrt{x^2 + y^2}} \\
    sin(\phi) = \frac{y}{\sqrt{x^2 + y^2}} \\
    cos(\theta) = \frac{\sqrt(x^2 + y^2)}{\sqrt{x^2 + y^2 + z^2}} \\
    sin(\theta) = \frac{z}{\sqrt{x^2 + y^2 + z^2}}
\end{gather}

\newpage

További számítás segédvektorok kiszámítása:

\begin{gather}
    V_x = \frac{x}{\sqrt{x^2 + y^2 + z^2}} *V = \frac{2}{\sqrt{44}} * 10 = \frac{10\sqrt{11}}{11} = 3.0151\\
    V_y = \frac{y}{\sqrt{x^2 + y^2 + z^2}} *V = \frac{-2}{\sqrt{44}} * 10 = -\frac{10\sqrt{11}}{11} = -3.0151\\\\
    V_z = \frac{z}{\sqrt{x^2 + y^2 + z^2}} *V = \frac{6}{\sqrt{44}} * 10 = \frac{30\sqrt{11}}{11} = 9.0453\\
\end{gather}

t idő múlva egy adott pillanatban koordináták meghatározása:

\begin{gather}
    x = X_0 + V_x * t = X_0 + \frac{X}{\sqrt{x^2 + y^2 + z^2}} * t = 5 + \frac{2}{\sqrt{44}} * 10 * t \\
    Y = y_0 + V_y * t = y_0 + \frac{y}{\sqrt{x^2 + y^2 + z^2}} * t = 4 - \frac{2}{\sqrt{44}} * 10 * t \\
    z = z_0 + V_z * t = z_0 + \frac{z}{\sqrt{x^2 + y^2 + z^2}} * t = 6 + \frac{6}{\sqrt{44}} * 10 * t
\end{gather}

\subsection{Az Ortodroma számítás, a föld alakját figyelembe véve}

Mivel gömbi geometria lényegesen eltér az euklideszi geometriától ezért a távolságszámításra használt matematikai képletek is eltérőek.
Az euklideszi geometriában a legrövidebb távolságot a két pontot összekötő egyenes,
a nem euklideszi geometriában a két pontot összekötő geodetikus vonal (gömb esetén főkör) mentén mérjük.

\paragraph{A legrövidebb út kiszámítása}
\begin{gather}
    d = arccos(sin\phi_1 * sin\phi_2 * cos\phi_1)* cos\phi_2 * cos\Delta \lambda) * R
\end{gather}

\paragraph{Irány kiszámítása}
Ahogy az főkörön haladunk az irányunk folyamatosan változik. Tehát amikor megérkezünk a célhoz, nem ugyanabban az irányba nézünk mint amikor elindultunk.
A drón szimulációban ez a telemetria adatokban is megfigyelhető.
\begin{gather*}
    \varphi = atan2(sin\Delta\lambda * cos\phi_{2}, cos\phi_{1} * sin\phi_{2} - sin\phi_{1} * cos\phi_{2} * cos\Delta\lambda)
\end{gather*}
ahol $\phi_{1}\lambda_{1}$ a kezdeti pont, és $\phi_{2}\lambda_{2}$ a vég pont és $\Delta\lambda$ pedig a különbség a hosszúsági fokokban.

\paragraph{Cél koordináták kiszámítása, ha ismerjük a megtett távolságot, irányt, és kezdő koordinátákat}
\begin{gather}
    \phi_{2} = arcsin(sin \phi_{1} * cos \delta + cos \phi_{1} * sin \delta * cos \theta)\\
    \lambda_{2} = \lambda_{1} + arctan2(sin\theta * sin\delta  * cos\phi_{1}, cos\delta - sin\phi_{1} * sin\phi_{2})
\end{gather}
ahol $\phi$ a szélességi fok, $\lambda$ a hosszúsági fok, $\theta$ az irány, $\delta$ a szögtávolság d/R és d a megtett távolság, R pedig a Föld sugara.

