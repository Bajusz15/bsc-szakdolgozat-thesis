\Chapter{Koncepció}

\Section{A fejezet célja}

Ez a fejezet még nem a saját eredményekkel foglalkozik, hanem bemutatja, mi a problémakör, milyen módszerekkel, milyeneredményeket sikerült elérni eddig másoknak.

A hivatkozások jelentős része ehhez a fejezethez szokott kötődni.
(Egy hivatkozás például így néz ki \cite{coombs1987markup}.)
Itt lehet bemutatni a hasonló alkalmazásokat.

\Section{Tartalom és felépítés}

A fejezet tartalma témától függően változhat. Az alábbiakat attól függően különböző arányban tartalmazhatják.
\begin{itemize}
\item Irodalomkutatás. Amennyiben a dolgozat egy módszer kidolgozására, kifejlesztésére irányul, akkor itt lehet részletesen végignézni (módszertani vagy időrendi bontásban), hogy az eddigiekben milyen eredmények születtek a témakörben.
\item Technológia. Mivel jellemzően kutatásról vagy szoftverfejlesztésről van szó, ezért annak a jellemző elemeit, technikai részleteit itt kell bemutatni.
Ez tehát egy módszeres bevezetés ahhoz, hogy ha valaki nem jártas a témakörben, akkor tudja, hogy a dolgozat milyen aktuálisan elérhető eredményeket, eszközöket használt fel.
\item Piackutatás. Bizonyos témáknál új termék vagy szolgáltatás kifejlesztése a cél.
Ekkor érdemes annak alaposan utánanézni, hogy aktuálisan milyen eszközök érhetők el a piacon.
Ez szoftverek esetében a hasonló alkalmazások bemutatását, táblázatos formában történő összehasonlítását jelentheti.
Szerepelhetnek képek és észrevételek a viszonyításként bemutatott alkalmazásokhoz.
\item Követelmény specifikáció. Külön szakaszban érdemes részletesen kitérni az elkészítendő alkalmazással kapcsolatos követelményekre.
Ehhez tartozhatnak forgatókönyvek (\textit{scenario}-k).
A szemléletesség kedvéért lehet hozzájuk képernyőkép vázlatokat is készíteni, vagy a használati eseteket más módon szemléltetni.
\end{itemize}

\Section{Hasonló rendszerek, megoldások áttekintése}

Az Amazonnak már működő szállítási rendszere van, mely 30 percen belül eljuttatja a csomagokat drónnal az ügyfelekhez.
Ezt a rendszert Prime Air-nek hívják \cite{prime-air}.
Először 2016-ban próbálták ki, azóta még nem sikerült bevezetni különböző jogi korlátozások miatt.
Az Amazon felhőszolgáltásai között is található olyan szolgáltatás amellyel könnyedén kezelhetjük az IoT eszközöket, jelen esetben a drónokat.
Ilyen az IoT Core vagy az IoT Things Graph amin grafikus felületen összeköthetjük és beállíthatjuk az eszközeink közti kommunikációt.
A DHL is próbálkozott hasonló csomagkézbesítő megoldásokkal, ők egy úgynevezett Parcelcopter-el szállították a csomagokat.
Ám ők más problémát akarnak megoldani, a nagyon nehezen elérhető helyekre probálnak csomagokat szállítani.
A FlytBase \cite{flyt} vállalat már kínál szoftveres megoldásokat a automatizált drónok valós idejű figyelésére és irányítására, útvonal tervezésre valamint flotta menedzselésre.

\Section{A szállítási probléma absztrakt modellje}

A szállítási modellünk alapvetően úgy nézne ki, hogy a raktár és a kiszállítási hely közti távolság függvényében a csomagokat a raktárból drónok, vagy drónokkal felszerelt teherautók viszik ki.
A teherautók a kiszállítási hely közelében használnák a drónokat hogy kézbesítsék a drónokat.
A drónok 3G, 4G vagy 5G hálózaton kommunikálnának az adatközpontokkal.
Az adott hálózaton elérhető sávszélesség függvényében a drónok változtathatják az elküldött adatok mennyiségét és a küldés ütemezését, gyakoriságát.
A drónok repülése előre megtervezett, minél hatékonyabb kell hogy legyen de valós időben kell reagálniuk a helyzetekre, problémákra.
A drónokból a szállítás során keletkező nagymennyiségű adatot az adatközpontoknak valós időben hibamentesen és hatékonyan kell tudnia kezelni anomáliák nélkül. Ehhez nagyon fontos a terhelés megfelelő elosztása.

A drónon csomag szállítási folyamatáról a következőket feltételezzük:
\begin{itemize}
    \item Egy drón  véges akkumlátoridővel vagy üzemanyaggal rendelkezik és mindig vissza kell tudnia térnie egy töltőállomásra, amely lehet egy drónszállító teherautó vagy raktár.
    \item A csomag kézbesítés a csomag átvételére vonatkozó ideje elhanyagolható.
    \item Minden drón fel van szerelve megfelelő kommunikácós és helymeghatározó eszközökkel, valamint az kommunikál az adatközpontokkal.
    \item Feltételezzük hogy a drónok el vannak látva szenzorokkal hogy felismerjék az akadályokat és erről értesítsék az adatközpontokat.
\end{itemize}

\Section{A Go nyelv áttekintése }
A Go program nyelv, (sokszor Golangként emlegetik) egy nyílt forráskódú modern programozási nyelv melyet a Google fejlesztett ki.
A Google-nél olyan problémák adódtak, hogy a konkurrenciát nem tudták megfelelően kezelni a még eredetileg 1 processzoros számítógépekre kifejlesztett nyelvek, mint a Java, C++.
Persze azóta sokat fejlődött ezen nyelvek konkurrenciakezelése de nem tudnak versenyezni a Go-val ha a gépi és fejlesztői hatékonyságot is figyelembe vesszük.
Probléma volt az is, hogy ezeknek a nyelveknek nagyon nagy a fordítási idejük. A 2000-es évek végén ez azt jelentette hogy egy Java vagy C++ program a Google-nél 1 hét alatt fordult le,
csak hogy ki tudják próbálni.
A nyelvek bonyolultságával is baj volt, ahogy fejlődtek a nyelvek egyre nehezebb volt egy régi programot tovább fejleszteni.
Hogy ezeket a problémákat orvosolják, a Google a legjobb tervezőket hívta össze, hogy megalkossák a Go nyelvet.
A tervezők közt volt Ken Thompson, Rob Pike és Robert Griesemer. A végeredmény egy olyan nyelv amely kifejező, tömör, tiszta és hatékony.
Szintaktikailag a C-hez hasonlít de nagyon könnyen tanulható nyelv.
A legnagyobb újítás a konkurrencia mechanizmus volt.

A Go konkurrencia mechanizmusa megkönnyíti olyan programok írását amelyet a legtöbbet hozzák ki a többmagos és hálózaton összekötött gépekből.
Gyorsan lefordul gépi kódra de rendelkezik garbage collection-el és runtime reflection is van beépítve.
Gyors, statikus nyelv de úgy érezzük mintha egy dinamikus futás időben fordított nyelv lenne.
Tartalmaz race detectort is ami nagyon fontos egy ilyen nyelvben, mert fel tudja ismerni a konkurrens/párhuzamos programozásnál felmerülő gyakori problémákat.
Mivel nagyon jól kihasználja a gép erőforrásait és a hálózatot manapság a felhőben futó kódok nagyrésze Go.
Ez lehet valamilyen mikroszerviz, infrastruktúra kezelés vagy konténer virtualizáció, konténer kezelés.
Például a Docker és Kubernetes Go-ban lett kifejlesztve, a felhőszolgáltatóknál ez a két szoftver a szolgáltatások alapja.

Pár vállalat aki használja:
\begin{itemize}
    \item Google %todo: mire hasznája
    \item Uber %todo: mire hasznája
    \item Netflix %todo: mire hasznája
    \item Youtube %todo: mire hasznája
    \item Dropbox %todo: mire hasznája
    \item BBC %todo: mire hasznája
\end{itemize}

Az előnyökhöz persze hátrányok is tartoznak, az egyszerűség mellé nem férnek el a generikus függvények.
Maga a nyelv nem tartalmaz osztályokat, illetve nincs öröklődés.
Viszont az objektum orientált paradigmákat használja és így is kell benne programozni.
Tartalmaz interfészeket, a típusokhoz így a structhoz is tudunk hozzárendelni úgynevezett receiver függvényeket, amelyeket más nyelvekben metódusoknak nevezünk, mert a típus egy példányára tudjuk meghívni.


\Section{Amit csak említés szintjén érdemes szerepeltetni}

Az olvasóról annyit feltételezhetünk, hogy programozásban valamilyen szinten járatos, és a matematikai alapfogalmakkal sem ebben a dolgozatban kell megismertetni.
A speciális eszközök, programozási nyelvek, matematikai módszerekk és jelölések persze jó, hogy ha említésre kerülnek, de nem kell nagyon belemenni a közismertnek tekinthető dolgokba.
