\Chapter{Tervezés}

Itt kezdődik a dolgozat lényegi része, úgy értve, hogy a saját munka bemutatása.
Jellemzően ebben szerepelni szoktak blokkdiagramok, a program struktúrájával foglalkozó leírások.
Ehhez célszerű UML ábrákat (például osztály- és szekvenciadiagramokat) használni.

Amennyiben a dolgozat inkább kutatás jellegű, úgy itt lehet konkretizálni a kutatási módszertant, a kutatás tervezett lépéseit, az indoklást, hogy mit, miért és miért pont úgy érdemes csinálni, ahogyan az a későbbiekben majd részletezésre kerül.

Ebben a fejezetben az implementáció nem kell, hogy túl nagy szerepet kapjon.
Ez még csak a tervezési fázis.
(Nyilván ha olyan a téma, hogy magának az implementációnak a módjával foglalkozik, adott formális nyelvet mutat be, úgy a kódpéldákat már innen sem lehet kihagyni.)

\Section{Program struktúra}
A probléma szimulációja 3 programból áll. Egy program az adatközpont, egy program a drónok szerepét veszi fel, a harmadik program vezéreli a szimulációt,
tehát igazából csak paraméterek alapján konfigurálja a másik 2 program működését. Az első két programot mikroszervízes struktúrában használható lesz.\\
Az adatközpontot szimuláló program lesz a legbonyolultabb. Ennek a programnak tudnia kell hibamentesen, adatvesztés nélkül
feldolgozni és tárolnia az adatokat amiket a drónok küldenek, és vezérelni kell tudni a drónok indítását.
A szimulációban több féle protokolt és adattovábbításra képes formátumot hasonlítunk össze, így a programot úgy kell felépíteni,
hogy paraméterek alapján ki lehessen cserélni ezeket a végpontokat.
Erre megoldásként a Hexagonal architektúrát és DDD tervezési alapelvet fogom alkalmazni.
%TODO: Ide ezekről képet beszúrni
Így a program belülről kifelé rétegesen epül fel, interfaceket alkalmazva, úgy hogy a pontos implemetáció absztraktálva van, az csak az üzleti logikát ismerjük.
így a végpontok ha teljesítik az interface elvárásait, csak dependency injection-el kicseréljük a végpontot és minden ugyanúgy működik, ám teljesen más az implementáció.\\
A drónokat szimuláló program olyan adatokat fog generálni, amelyet egy valós drón generálna és ezeket az adatokat tovább fogja küldeni az adatközpontnak.
A megfelelő működéshez egy algoritmust kell írni, aminek az outputja nagyon hasonlít eredeti mintákhoz.

\Section{Táblázatok}

Táblázatokhoz a \texttt{table} környezetet ajánlott használni.
Erre egy minta \aref{tab:minta}. táblázat.
A hivatkozáshoz az egyedi \texttt{label} értéke konvenció szerint \texttt{tab:} prefixszel kezdődik.

\begin{table}[h]
\centering
\caption{Minta táblázat. A táblázat felirata a táblázat felett kell legyen!}
\label{tab:minta}
\begin{tabular}{l|c|c|}
a & b & c \\
\hline
1 & 2 & 3 \\
4 & 5 & 6 \\
\hline
\end{tabular}
\end{table}

\Section{Ábrák}

Ábrákat a \texttt{figure} környezettel lehet használni.
A használatára egy példa \aref{fig:cimer}. ábrán látható.
Az \texttt{includegraphics} parancsba 
Az ábrák felirata az ábra alatt kell legyen.
Az ábrák hivatkozásához használt nevet konvenció szerint \texttt{fig:}-el célszerű kezdeni.

\begin{figure}[h]
\centering
\includegraphics[scale=0.3]{images/me_logo.png}
\caption{A Miskolci Egyetem címere.}
\label{fig:cimer}
\end{figure}

\Section{További környezetek}

A matematikai témájú dolgozatokban szükség lehet tételek és bizonyításaik megadására.
Ehhez szintén vannak készen elérhető környezetek.

\begin{definition}
Ez egy definíció
\end{definition}

\begin{lemma}
Ez egy lemma
\end{lemma}

\begin{theorem}
Ez egy tétel
\end{theorem}

\begin{proof}
Ez egy bizonyítás
\end{proof}

\begin{corollary}
Ez egy tétel
\end{corollary}

\begin{remark}
Ez egy megjegyzés
\end{remark}

\begin{example}
Ez egy példa
\end{example}
