\pagestyle{empty}

\noindent \textbf{\Large CD Használati útmutató}
%
%\vskip 1cm
%
%Ennek a címe lehet például \textit{A mellékelt CD tartalma} vagy \textit{Adathordozó használati útmutató} is.
%
%Ez jellemzően csak egy fél-egy oldalas leírás.
%Arra szolgál, hogy ha valaki kézhez kapja a szakdolgozathoz tartozó CD-t, akkor tudja, hogy mi hol van rajta.
%Jellemzően elég csak felsorolni, hogy milyen jegyzékek vannak, és azokban mi található.
%Az elkészített programok telepítéséhez, futtatásához tartozó instrukciók kerülhetnek ide.


A CD tartalma:
\begin{itemize}
\item dolgozat jegyzék
\item drone-delivery jegyzék
\end{itemize}
A dolgozat.pdf tartalmazza a szakdolgozatot, a drone-delivery jegyzék a programokat és a hozzá tartozó rendszert, méréseket.
A dolgozatban van részletezve a programok telepítése és használata.
A mérésekhez található html dokumentumokat meg lehet tekinteni önállóan.

