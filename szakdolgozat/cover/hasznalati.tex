\pagestyle{empty}

\noindent \textbf{\Large CD Használati útmutató}
%
%\vskip 1cm
%
%Ennek a címe lehet például \textit{A mellékelt CD tartalma} vagy \textit{Adathordozó használati útmutató} is.
%
%Ez jellemzően csak egy fél-egy oldalas leírás.
%Arra szolgál, hogy ha valaki kézhez kapja a szakdolgozathoz tartozó CD-t, akkor tudja, hogy mi hol van rajta.
%Jellemzően elég csak felsorolni, hogy milyen jegyzékek vannak, és azokban mi található.
%Az elkészített programok telepítéséhez, futtatásához tartozó instrukciók kerülhetnek ide.

\bigskip

\noindent A CD tartalma:
\begin{itemize}
    \item \textbf{\textit{dolgozat} jegyzék}:
    Ez tartalmazza magáta szakdolgozatot \textit{.tex} formátumban, és a hozzátartozó képeket, stílusokat.
    \item \textbf{\textit{drone-delivery} jegyzék}:
    Ebben a jegyzékben a programokat és a hozzá tartozó rendszert, méréseket találjuk.
    A dolgozatban van részletezve a programok telepítése és használata.
    A mérések adatait tartalmazó HTML dokumentumokat meg lehet tekinteni önállóan, a \textit{benchmark} jegyzéken belül.
    \item \textbf{\textit{szakdolgozat.pdf}}:
    A szakdolgozat \textit{.pdf} formátumban.
\end{itemize}



